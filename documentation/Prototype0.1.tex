\documentclass[10pt,a4paper]{article}
\usepackage[utf8]{inputenc}
\usepackage{amsmath}
\usepackage{amsfonts}
\usepackage{amssymb}
\title{\textbf{Anleitung zum Prototypen \texttt{lylly} 0.1}}
\author{Julian Rosemann, Nicklas Linz}
\begin{document}
\maketitle
\section{Konzept}
Dieser Prototyp dient zur Vorstellung von Konzepten, die spätere Versionen implementieren werden. Die wichtigsten Features sind im Kern umgesetzt und können getestet werden – diese Version ist aber nicht geeignet um tatsächlich von Endbenutzern eingesetzt zu werden. Zum einen wegen ihrer Instabilität, zum anderen weil das User-Interface noch keine effektive Nutzung ermöglicht.

Die Android App \texttt{lylly} soll der Anwenderin helfen ihre Zeit bewusst auf vordefinierte Ziele zu verwenden. Die Anwenderin muss hierfür in die App verschiedene Kategorien eintragen, in denen sie ihre aufgebrachte Zeit genauer kontrollieren bzw. steuern möchte.

In diesen Kategorien ist es ihr nun möglich Zeitintervalle zu definieren und anzugeben, wie viel Zeit sie für diese Kategorie in dem Zeitraum mindestens und höchstens investieren möchte. Zusätzlich kann die Nutzerin Angaben darüber machen, wie sie plant die Arbeit über die im Intervall enthaltenen Tage zu verteilen. Hierdurch kann die App der Nutzerin Vorschläge machen, wie viel Zeit sie am heutigen Tag für eine bestimmte Kategorie aufbringen sollte. Sollte die Nutzerin den Vorschlägen nicht nachgehen, werden die Vorgaben für die nächsten Tage so angepasst, das alle Ziele immer noch erreichbar sind.

Des Weiteren kann die Nutzerin Aufgaben definieren, welche jeweils einer Kategorie zugeordnet sind. Immer wenn die Benutzerin eine Aufgabe bearbeitet, wird die Zeit gestoppt und mit der Zielvorgabe der jeweiligen Kategorie verrechnet. Die Aufgaben können nach Kategorie gefiltert angezeigt werden, sodass die Anwenderin gezielt nach Aufgaben suchen kann, die ihr helfen ihre Zeitvorgaben zu erfüllen.

Grundsätzlich wird der Nutzerin empfohlen die Vorgaben der App zu befolgen, beziehungsweiße bei Nichtbefolgung ihre Verteilung so anzupassen, dass sie diese in der Zukunft erfüllen kann.

\section{Funktionalität}
Die App gliedert sich in den TaskOrganizer, den ProspectOrganizer und den TagOrganizer in denen die Aufgaben, Ziele beziehungsweise Kategorien verwaltet werden.

\subsection{TagOrganizer}
Hier können mit dem "+"-Button neue Kategorien hinzugefügt werden und bestehende modifiziert beziehungsweise gelöscht werden in dem man auf sie tippt.

\subsection{ProspectOrganizer}
Dies ist der Bildschirm auf dem bereits existierende Ziele angezeigt werden und bearbeitet werden können. Des weiteren können mithilfe des "+"-Buttons neue angelegt werden.

Bei jedem Ziel wird ganz links angezeigt, wie viel Zeit bereits darauf verwendet wurde. Darüber ist zu sehen wie viel Zeit noch zur Mindestforderung fehlt. Unten wie viel Zeit noch zur Höchstforderung fehlt. Dann kommt der Name der Kategorie und rechts am Rand stehen jeweils Start- und Enddatum übereinander.
Wenn der Name durchgestrichen ist, ist das Ziel verworfen.
\subsubsection{Ziele bearbeiten}
Ein Ziel besteht aus den folgenden Elementen:
\begin{itemize}
	\item \textsf{Kategorie:} die zugeordnete Kategorie
	\item \textsf{Startdatum:} am Anfang dieses Tages beginnt der Abrechnungszeitraum
	\item \textsf{Enddatum:} am Anfang dieses Tages endet der Abrechnungszeitraum 
	\item \textsf{Gewichte:} siehe unten
	\item \textsf{Mindestaufwand:} der Aufwand in Stunden und Minuten der insgesamt mindestens betrieben werden soll
	\item \textsf{Höchstaufwand:} der Aufwand in Stunden und Minuten der insgesamt höchstens betrieben werden soll
\end{itemize}
Die Gewichte sind eine Folge von \textsf{genau} so vielen Ziffern wie das Intervall (= Enddatum - Startdatum) Tage hat. Diese Ziffern geben an wie viel Zeit an den Tagen relativ zueinander in das Ziel investiert werden soll. Die Gewichtsangaben "1-1-2" und "2-2-4" bedeuten also genau dasselbe – am dritten Tag soll doppelt so viel Zeit investiert werden wie am ersten oder zweiten. "2-2-3" hingegen wäre eine etwas ausgewogenere Verteilung.

Ziele können - wenn sie bereits laufen - nur noch eingeschränkt bearbeitet werden: Gewichte können immer bearbeitet werden, alles andere kann nicht mehr bearbeitet werden, wenn das Startdatum erreicht wurde. Ziele können allerdings immer verworfen werden.

\subsection{TaskOrganizer}
Auf diesem Bildschirm wird die Nutzerin die meiste Zeit verbringen. Der TaskOrganizer wird in dieser Dokumentation zuletzt aufgeführt, da das sinnvolle Nutzen dieses Bildschirms das vorherige Erstellen der Kategorien und Ziele in TagOrganizer und ProspectOrganizer erfordert.

Der Bildschirm ist in zwei Abschnitte gegliedert:
\begin{itemize}
	\item die Liste von Kategorien mit je einem Vorschlag für Mindest- und Höchstwert der \textit{heute} zu investierenden Zeit pro Kategorie
	\item die Liste von Aufgaben, benutzerdefiniert gefiltert
\end{itemize}
In der Liste der Kategorien werden für jede Kategorie die Zeiten ähnlich angezeigt, wie bei den Zielen und es gibt je einen Button zum Ein- beziehungsweise Ausblenden aller Aufgaben dieser Kategorie. Wenn alle Buttons auf "Aus" geschaltet sind, werden alle Aufgaben angezeigt. Des weiteren ändert sich die Farbe der Kategorie, wenn Mindestwert (grün) oder Höchstwert (orange) erreicht wird. 

Aufgaben können durch den "+"-Button hinzugefügt werden und durch antippen bearbeitet werden. Außerdem gibt es für jede Aufgabe einen "Play/Pause"-Button sowie einen "Done"-Button. Diese sollten benutzt werden um der App zu signalisieren, wann die Aufgabe bearbeitet wird oder erledigt ist. Der Zähler an der Aufgabe läuft aufwärts, wenn die Aufgabe in Bearbeitung ist. Beim Bearbeiten einer Aufgabe wird auch die Liste von Intervallen angezeigt, in denen die Aufgabe bearbeitet wurde. Diese können dort auch bearbeitet oder neu hinzugefügt werden.

\section{Zukünftige Features}
Im folgenden werden einige potentielle Features für zukünftige Versionen aufgelistet. 
\subsection{Neue Konzepte}
\begin{itemize}
	\item Mehr Einstellungen für Aufgaben
	\begin{itemize}
		\item Fälligkeitsdatum
		\item Wichtigkeit
		\item Geschätzte Dauer (?)
	\end{itemize}
	\item Verbessertes Modell für Gewichte bzw. deren Eingabe (eventuell Gewichte pro Woche und pro Wochentag)
	\item Wiederholende Aufgaben
	\item Wiederholende Ziele
	\item Ineinander verschachtelte Aufgaben
	\item Ineinander verschachtelte Ziele
	\item Ineinander verschachtelte Kategorien
	\item Warnungen bei Unerreichbarkeit, z.B. ein Tag mit über 24 Stunden Zielen insgesamt
	\item Belohnungssystem, z.B. für wiederholtes Erreichen der Tagesziele
\end{itemize}
\subsection{User-Interface Verbesserungen}
\begin{itemize}
	\item Intuitivere Eingabe von Gewichten
	\item Möglichkeit Gewichte verschiedener Kategorien gleichzeitig zu ändern
\end{itemize}
\end{document}